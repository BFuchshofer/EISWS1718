\section{MCI Inhalte:}
\label{sec:MCI Inhalte}

\subsection{Einleitung}
\label{sec:Einleitung}

Im Folgenden beschäftigen wir uns mit den Mensch-Computer Interaktion-Elementen
des interaktiven Softwareprojektes.

\subsection{Stakeholderanalyse}
\label{sec:Stakeholderanalyse}

Um einen ersten Überblick über die möglichen Stakeholder des Systems zu bekommen
haben wir im Rahmen einer Stakeholderanalyse potenzielle Benutzer und 
Interessenten des Systems analysiert. 
\begin{itemize}
	\item Lehrkräfte
	\item Lerner
	\item Wissenschaftliche Mitarbeiter
	\item Lehreinrichtungsverwaltung
	\item Administratoren
	\item Angestellte
	\item Lehreinrichtung
\end{itemize}

Eine ausführliche Auflistung aller identifizierten Stakeholder mit ihren
Eigenschaften und Bezügen zum System befindet sich im ( siehe \ref{tabelle_Stakeholderanalyse}).

\section{Vorgehensmodelle}
\label{sec:Vorgehensmodelle}

\subsection{Einleitung}
\label{sec:Vorgehensmodelle_Einleitung}

Das anzuwendende Vorgehensmodell in diesem Projekt behandelt im Idealfall
die Charakteristiken des Projektfeldes. Dazu zählen \zB die Projektziele,
Aufgaben und organisationale Anforderungen die gegeben sein müssen, um das
Projekt effektiv bearbeiten zu können. Im Folgenden soll deduktiv auf ein
passendes Vorgehensmodell für dieses Projekt geschlossen werden, welches im
Entwicklungsprozess angewendet wird.

\subsection{Auswahl eines Vorgehensmodell}
\label{sec:Auswahl_eines_Vorgehensmodell}
Da das zu entwickelnde System dem Benutzer eine Möglichkeit bieten muss ein
ganz bestimmtes Ziel zu erreichen, dass innerhalb der Anwendungsdomäne immer
das selbe ist, liegt es nahe, das in der Entwicklung ein besonderer Fokus auf
die Modellierung der Aufgaben eines Benutzers gelegt werden muss. Die Aufgaben
deren Erfüllung das System gewährleisten muss, beschränken sich auf ein
übergeordnetes Ziel, dem Suchen eines Raumes. Die Benutzer und ihre
Eigenschaften haben zwar eine gewisse Relevanz, müssen aber nicht wie \zB im
\textit{[User-Centred Design](Buch) von Norman und Draper} im Fokus des gesamten
Entwicklungsprozzesses liegen. Die Aufführung aller potenziellen Benutzer des
Systems schränkt sich, wenn man das Anwendungsumfeld betrachtet, stark ein.
Das führt uns zu dem Schluss, dass eine ausführliche Benutzermodellierung in
diesem Kontext nicht zwingend notwendig ist. Aus diesem Grund sind auch
Vorgehensmodelle die eine starke Einbindung von Benutzern oder Vertretern
dieser beinhalten nicht relevant für diesen Projektkontext.
Dazu zählen \zB das \textit{[Scenario Based Usability Engineering](Buch) Vorgehen von Mary Beth Rosson und John M. Carrol},
das mittels einfach zu verstehender Szenarien Benutzern die Möglichkeit bietet
effektiv am Entwicklungsprozess teilhaben zu können. Daher sollte der Fokus
unseres Projektes eher auf die möglichst effektive Nutzung unseres Systems
gelegt werden. Dadurch soll sichergestellt werden, dass der Benutzer oder die
Benutzergruppe durch die Verwendung unseres Systems hinsichtlich seines
Hauptziels nicht behindert sonder viel mehr unterstützt wird.

\subsection{Fazit - Vorgehensmodell}
\label{sec:Fazit_Vorgehensmodell}

Da in unserem Projekt weder die Einbindung von Benutzern in den
Entwicklungsprozess noch die ausführliche Benutzermodellierung in einem
benutzerzentriertem Vorgehen ein besonders großen Wert hat, haben wir uns für
das \textit{[Usage-Centred Design](Buch) von Constantine und Lockwood}
entschieden. Dieses Vorgehensmodell legt einen besonderen Wert auf die
Anwendung des Systems, also die Fokussierung auf den Aufgabenbereich einer
Anwendung. Desweiteren ist es durch seine Skalierbarkeit ein ideales
Vorgehensmodell für ein Projekt dieser Größe.

\section{Benutzermodellierung}
\label{sec:Benutzermodellierung}

\subsection{Einleitung}
\label{sec:Benutzermodellierung_Einleitung}

Obwohl im \textit{Usage-Centred Design von \citet{softwareForUse}}
die Benutzermodellierung keinen sehr hohen Detaillierungsgrad hat, ist dennoch
eine Übersicht über Benutzergruppen wichtig um ein effektiv funktionierendes
System zu entwickeln. Aus den Benutzergruppen lassen sich benötigte Aufgaben
ableiten, die das System abdecken muss. Da wir konkrete Benutzer nur an
bestimmten Stellen in den Entwicklungsprozess einbeziehen werden, wenden wir
die Benutzermodellierung der User Roles \textit{\citep[Kapitel~4]{softwareForUse}} an die es uns erlaubt,
Benutzergruppen mit ihren Erwartungen an das System zu modellieren. Dabei wird
darauf Rücksicht genommen was diese Benutzer-Rollen im Bezug auf ihre
Aufgaben innerhalb des Systems darstellen.

\subsection{User Roles}
\label{sec:User_Roles}

Anhand der durchgeführten Stakeholderanalyse lassen sich bereits potenzielle,
aktive Benutzer ableiten. Um diese konkret in Rollen unterteilen zu können,
haben wir die Erkenntnisse aus der Stakeholderanalyse dafür genutzt in einem
Brainstorming Rollennamen zu identifizieren die ihre Aufgabe und ihren Bezug
zum System widerspiegeln.
Um die Rollen kurz charakteristisch zu beschreiben, haben wir zusätzlich eine
Kurzbeschreibung der einzelnen Rollen angefertigt, die die Unterschiede
deutlich machen soll. Wir halten das für notwendig, da die einzelnen Rollen
sich teilweise nur marginal voneinander unterscheiden.
Dabei sind wir auf folgende Ergebnisse gekommen:

\begin{itemize}
	\item SilentWorkingRoomSearcher
	\begin{itemize}
		\item sucht einen stillen Arbeitsplatz.
	\end{itemize}
	\item CasualSingleRoomSearcher
	\begin{itemize}
		\item sucht einen Arbeitsraum in dem in einer Gruppe gearbeitet werden kann.
	\end{itemize}
	\item CasualMultiRoomSearcher
	\begin{itemize}
		\item sucht mehrere Arbeitsräume in denen in Gruppen gearbeitet werden kann.
	\end{itemize}
	\item CasualRoomWithEquipmentSearcher
	\begin{itemize}
		\item sucht einen Arbeitsraum der spezielles Equipment beinhaltet, das die Benutzer zum arbeiten benötigt.
	\end{itemize}
	\item RoomWithProfessionalEquipmentSearcher
	\begin{itemize}
		\item sucht einen Arbeitsraum der spezielles Equipment beinhaltet, das die Benutzer zum arbeiten benötigen.
	\end{itemize}
	\item SpecificRoomBooker
	\begin{itemize}
		\item Benutzer möchte einen bestimmten Raum, ohne vorherige Reservierung buchen.
	\end{itemize}
	\item RoomScheduler
	\begin{itemize}
		\item möchte den Status eines Raumes abfragen.
		\item möchte einen oder mehrere Räume für einen Zeitraum in der Zukunft reservieren/buchen.
	\end{itemize}
	\item ServerAdministrator
	\begin{itemize}
		\item möchte den Server des Systems verwalten und aktualisieren.
	\end{itemize}
	\item DBAdministrator
	\begin{itemize}
		\item möchte die Datenhaltung des Systems verwalten und aktualisieren.
	\end{itemize}
	
\end{itemize}

Während der Ermittlung der Rollennamen sind uns Dopplungen in den Bedürfnissen
der Rollen aufgefallen, die wir im nächsten Schritt eliminieren wollen.
Dazu haben wir die Rollen neu sortiert und versucht Gruppen zu bilden die
ähnliche Interessen besitzen.
Dabei ließen sich drei Gruppen mit jeweils unterschiedlichen Spezialisierungen finden.

\begin{itemize}
	\item FreeRoomSearcher als Gruppe von Benutzern die daran interessiert sind einen Raum zu suchen in dem Arbeiten verrichtet werden können.
	\begin{itemize}
		\item SilentWorkingRoomSearcher
    	\item CasualSingleRoomSearcher
   		\item CasualMultiRoomSearcher
    	\item CasualRoomWithEquipmentSearcher
    	\item RoomWithProfessionalEquipmentSearcher
    	\item SpecificRoomBooker
    	\item RoomScheduler
	\end{itemize}
	\item Administrator als Gruppe von Benutzern, die sich um die Verwaltung des Systems kümmert.
	\begin{itemize}
	    \item ServerAdministrator
    	\item DBAdministrator
	\end{itemize}
	\item RoomStatusChecker als Gruppe von Benutzern, die Informationen über den aktuellen oder zukünftigen Status eines Raumes benötigen.
	\begin{itemize}
		\item SpecificRoomBooker
		\item RoomScheduler
		\item SpecificRoomBlocker
	\end{itemize}
\end{itemize}

Eine genaue Auflistung der \textit{Role Models} mit ihren Interessen und Bedürfnissen
wie im Buch \textit{Software for use} beschrieben, findet sich im \ref{[Anhang](tabelle_User_Roles.md)}.

Auch in dieser Auflistung befinden sich noch einige Rollen doppelt, was damit
zusammenhängt, dass bestimmte Rollen Eigenschaften besitzen, die die
Bedürfnisse von mehreren Rollen widerspiegeln. Sie sind dabei aber nicht als
zwei unterschiedliche Rollen zu verstehen, sondern als Spezialisierungen oder
Erweiterungen einer bestehenden Rolle. Um diese Abhängigkeiten deutlicher zu
machen, haben wir im nächsten Schritt zusätzlich zu den \textit{User Roles} ein
Schaubild erstellt. Diese User Role Map \textit{\citep[Kapitel~4]{softwareForUse}} dient uns dazu,
einen Zusammenhang der Rollen untereinander und zum System zu gewähren, damit
ein genereller Überblick geschaffen wird der uns hilft im Entwicklungsprozess
die Verknüpfungen unterschiedlicher Aspekte nicht aus den Augen zu verlieren.
Die \textit{User Role Map} befindet sich zur besseren Lesbarkeit im \ref{[Anhang](user_role_map.png)}.

       
\subsection{Focal Roles}
\label{sec:Focal_Roles}

Die wichtigsten \textit{User Roles} in unserem System sind unserer Meinung nach die
Rollen \textbf{CasualSingleRoomSearcher}, der nur einen normalen Raum zum
Arbeiten benötigt, und der \textbf{CasualRoomWithEquipmentSearcher} der einen
Raum sucht, der ein bestimmtes Equipment beinhaltet. Da diese Rollen die
allgemeinsten, aber wohl auch die häufigsten Rollen in unserem System sein
werden, haben wir diese als \textit{Focal Roles} identifiziert.
Die Focal Roles \citep[Kapitel~4]{softwareForUse} sind laut \textit{\citep{softwareForUse}} die
Rollen, die innerhalb eines Systems die typischsten und wichtigsten
Benutzergruppen darstellen. Innerhalb unseres Nutzungskontextes sind diese
beiden Rollen am häufigsten vertreten und außerdem ausschlaggebend für die
Zielerreichung des eigentlichen Nutzungsproblems. Im Entwicklungsprozess haben
die Bedürfnisse der \textit{Focal Roles} einen wichtigen Anteil an der Gestaltung des
User Interfaces, weswegen sie im Verlauf der Gestaltung immer wieder fokussiert
betrachtet werden.

\begin{quote}
	Focal roles are those few user roles judged to be the most common or typical or that 
	are deemed particularly important fromsomeother perspective.\\
	\citep[Seite 83]{softwareForUse}
\end{quote}

\subsection{Fazit - Benutzermodellierung}
\label{sec:Fazit_Benutzermodellierung}

Die identifizierten \textit{User Roles} aus unserer Analysearbeit werden im Verlauf
des Entwicklungsprojektes dazu genutzt werden die Aufgaben der Benutzer zu
spezifizieren und die Anforderungen an das System zu erheben.
Außerdem werden die \textit{Focal Roles} dazu genutzt werden den Entwicklungsprozess
des User Interface mit einer fokussierten Betrachtungsweise zu unterstützen.
\citep{softwareForUse} schlagen als weitere Ergänzung der
Benutzermodellierung das Entwerfen von Structured Role Models \citep[Kapitel~4]{softwareForUse} vor.
Wir sind der Meinung das dieser Modellierungsentwurf für dieses Projekt nicht
benötigt wird, da eine ausführliche Unterscheidung der Benutzer in den
\textit{User Roles} bereits vorgenommen wurde. Die Unterschiede der
identifizierten \textit{User Roles} sind minimal, und unterscheiden sich nur in
einigen wenigen Punkten, weswegen es unserer Meinung nach nicht notwendig ist
diese weiter zu verfeinern. \citep{softwareForUse} geben weiterhin an,
dass der Modellierungsschritt der \citep[Kapitel 4]{softwareForUse} für kleine
Projekte oder Projekten mit kurzer Zeitspanne vernachlässigt werden kann
\citep[Seite 89, 2. Absatz, 3. Zeile]{softwareForUse}, was uns in unserer Meinung zusätzlich bestätigt.



\section{Benutzungsmodellierung}
\label{sec:Benutzungsmodellierung}

\subsection{Einleitung}
\label{sec:Benutzungsmodellierung_Einleitung}

Da im Vorgehen des \textit{Usage-Centred Design} der Fokus auf dem
Identifizieren und Verstehen von Aufgaben der Benutzer liegt, haben wir mehrere
Modellierungsstufen angewandt um die Aufgaben der Benutzer zu erarbeiten.
Dabei sind wir durch immer weiteres und feineres Vorgehen auf die schlussendlich
benötigten Aufgaben gestoßen, dessen Erfüllung das System gewährleisten muss.

\subsection{Deskriptive Aufgabenmodellierung}
\label{sec:Deskriptive_Aufgabenmodellierung}
Um einen Überblick über die Aufgaben der Benutzer zu erhalten, haben wir
zuallererst eine Auflistung der Aufgaben vorgenommen die im aktuellen
Nutzungskontext für das Erreichen des Nutzungsziels benötigt werden. 
Da kein aktives System vorliegt, das diese Aufgaben bearbeitet, sind die 
meisten der Aufgaben physischer Natur ohne ein technisches System. 
Die Aufgaben lauten wie folgt:

\begin{itemize}
	\item Freiern Raum suchen
	\begin{itemize}
		\item Gebäudeplan studieren
		\item Gebäude/Stockwerk/Gang herausfinden in dem sich mögliche Arbeitsräume befinden
		\item (optional) Hausmeister/Info/Sekretariat nach freien Räumen fragen
		\item (optional) mögliche bereits bekannte freie Räume aus dem Gedächtnis abrufen
		\item (optional) Raumbelegungspläne, sofern vorhanden, Online prüfen
		\item zum freien Raum bewegen
		\item (optional) Raumbelegungsplan vor der Tür auf Belegung überprüfen
		\item Tür öffnen (sofern möglich) um zu Prüfen ob der Raum frei ist
			\begin{itemize}
				\item im Raum nach benötigtem Equipment schauen
			\end{itemize}
		\item falls der Raum belegt ist, iterieren und \ggfs Gebäude/Stockwerk/Gang wechseln
		\item falls Equipment nicht vorhanden ist, iterieren und \ggfs Gebäude/Stockwerk/Gang wechseln
	\end{itemize}
	\item Freien Raum belegen
	\begin{itemize}
		\item in den Raum begeben und anfangen seine geplante Arbeit zu verrichten
		\item \ggfs Raum für wichtige Veranstaltungen wieder freigeben
		\begin{itemize}
			\item erneut nach einem Raum suchen
		\end{itemize}
	\end{itemize}
\end{itemize}

Der Prozess der effektiven Raumsuche gestaltet sich im aktuellen Kontext
relativ schwierig, da nur mit Glück das schnelle Finden eines passenden Raumes
ermöglicht wird. Im schlimmsten Fall wird bei der Raumsuche viel Zeit
verschwendet die anderweitig besser genutzt werden kann. Außerdem gibt es
weder eine Garantie das ein Raum frei ist wenn man ihn aufsucht, noch das man
überhaupt einen freien Raum in absehbarer Zeit finden kann.

\subsection{Präskriptive Aufgabenmodellierung}
\label{sec:Präskriptive_Aufgabenmodellierung}

Um im nächsten Schritt die präskriptiven Aufgaben der verschiedenen
Benutzergruppen zu erschließen, haben wir unter Zuhilfenahme der
identifizierten \textit{user roles} ein Brainstorming durchgeführt, um einen
ersten Überblick über mögliche benötigte Aufgaben \bzw Bedürfnisse der Benutzer
zu erhalten.
Dabei haben wir uns folgende Fragen gestellt:

\begin{itemize}
	\item What are users in this role trying to accomplish?
	\item To fulfill this role, what do users need to be able to do?
	\item What capabilities are required to support whatever users in this role need to accomplish?\\
	\citep[Seite 116,Zeile 7-9]{softwareForUse}
\end{itemize}


Auszugsweise im folgenden ein paar exemplarische identifizierte Aufgaben die
sich aus dem Brainstorming ergeben haben. Die vollständige Liste der Aufgaben
aus dem Brainstorming finden sich im \ref{[Anhang](tabelle_Use_Cases.md)}.

\begin{itemize}
	\item Der Benutzer sucht einen ruhigen Raum in dem er Arbeiten kann.
	\item Der Benutzer sucht einen freien Raum mit bestimmtem Equipment den er nutzen kann.
	\item Der Administrator möchte einen bestehenden Raum im System mit neuen Informationen anreichern.
\end{itemize}

Beim Brainstorming haben wir festgestellt, dass einige der Aufgaben  starke
Ähnlichkeiten besitzen was den Aufbau betrifft. So sind der Hauptteil der
Aufgaben dafür zuständig einen Raum zu suchen. Lediglich die Art des Raumes die
der Benutzer zum Arbeiten benötigt unterscheidet die Aufgaben. So lassen sich
grobe Unterscheidungen bei den Aufgaben treffen die wir im späteren Verlauf
genauer beschreiben und darstellen werden.

Da diese Aufgaben teilweise ungenau formuliert, \bzw noch Mehrdeutigkeiten
zuließen, haben wir die identifizierten Aufgaben in \textit{Essential Use Cases}
formuliert, die die wichtigsten Bedürfnisse der Benutzer und die Reaktion des
Systems auf diese Bedürfnisse widerspiegeln \ref{[tabelle_Use_Cases](tabelle_Use_Cases.md)}.
Unserer Meinung nach sind Use Cases ein valides und sinnvolles Mittel um eine
Aufgaben-zentrierte Softwarelösung zu entwickeln.
Die erarbeiteten \textit{Essential Use Cases} sind laut Definition \citep[Kapitel~5]{softwareForUse}
technologieunabhängig und rein auf die Benutzerabsicht und die Reaktion des
Systems bezogen, was es uns im späteren Entwicklungsprozess einfacher macht
technologiebezogene Gestaltungslösungen zu entwerfen.

Um die erarbeiteten und benötigten Aufgaben evaluieren zu können, haben wir
uns dazu entschlossen diese noch in der Form des \textit{Narrative Concrete Use Cases}
zu formulieren \ref{[tabelle_Use_Cases](tabelle_Use_Cases.md)} \citep[Kapitel 5]{softwareForUse}.
Diese Use Cases wurden von uns in einer narrativen und verständlichen, in der
Anwendungsdomäne befindlichen Sprache verfasst, um sie in einem
Evaluationsprozess potenziellen Benutzern vorzustellen.
Da narrative Texte für den Benutzer einfacher zu verstehen sind, bietet sich uns
damit die Möglichkeit ein paar Benutzer mit den von uns ermittelten Aufgaben zu
konfrontieren und ihre Meinung und eventuelle Ergänzungen einzuholen.
Da die \textit{Conrecte Use Cases} die Nutzung des Systems ausführlicher beschreiben
als die \textit{Essential Use Cases}, wird es für den Benutzer einfacher die
Handlungen nachzuvollziehen.
Das Feedback der Benutzer wird dazu genutzt die Aufgaben zu überarbeiten und
zu verbessern. 


\subsection{Focal Use Cases}
\label{sec:Focal_Use_Cases}
Wie schon in den \textit{Focal Roles} in der Benutzermodellierung, gibt es
\textit{Focal Use Cases} die in unserem Fall die wichtigsten \bzw am häufigsten
vorkommenden Anwendungsfälle darstellen. Dabei haben wir auf die \textit{Focal Roles}
zurückgegriffen und uns überlegt welche Aufgaben diese \textit{Focal Roles}
innerhalb des Systems erfüllen müssen.
Unserer Meinung nach sind das die Aufgaben \textbf{searchingRoomWithSpecificEquipment}
und \textbf{searchingRoomForSeveralPersons}. Diese Anwendungsfälle haben in unserem
System die größte Relevanz da die Funktionalitäten hinter diesen Aufgaben den
Grundstein aus Benutzersicht für dieses Projekt legen. 


\subsection{Iterative Evaluation der Aufgaben}
\label{Iterative_Evaluation_der_Aufgaben}

Um die ermittelten Aufgaben zu überprüfen, haben wir ein Gespräch mit
2 Studenten geführt die sich im Umfeld der Anwendungsdomäne befinden und somit
potenzielle Benutzer des Systems darstellen. Wir haben ihnen die von uns
aufgestellten Aufgaben in der Form der \textit{Narrativ Conrete Use Cases} vorgestellt
und sind mit ihnen folgende Fragen für jeden Use Case durchgegangen:

\begin{itemize}
	\item Können Sie sich vorstellen diese Aufgaben innerhalb des Problemkontextes durchzuführen?
	\item Welche Vorteile sehen Sie an diesem Anwendungsfall?
	\item Welche Nachteile sehen Sie an diesem Anwendungsfall?
\end{itemize}

Nachdem alle Use Cases besprochen wurden und Feedback dokumentiert wurde, haben
wir dem Benutzer noch folgende zusätzliche Frage gestellt:

\begin{itemize}
	\item Können Sie sich weitere Anwendungsfälle vorstellen die bisher nicht erwähnt worden sind?
\end{itemize}

Wir haben dabei in Kooperation mit den Befragten Benutzern festgestellt das es
gewisse Risiken für die Umsetzung des Systems gibt. Zum Beispiel ist nicht
gewährleistet das ein Raum auch tatsächlich frei ist wenn der Benutzer ihn vom
System vorgeschlagen bekommt und anschließend aufgesucht hat. Es wird also eine
Art Reservierungsfunktion benötigt die einen Raum für andere Benutzer temporär
als nicht verfügbar kennzeichnet, bis der tatsächliche Benutzer diesen
Raum körperlich erreicht hat. Zusätzlich  wird die Möglichkeit der vorzeitigen
Stornierung eines Raumes benötigt, falls sich der Benutzer spontan entschließt
den Raum nicht mehr zu benutzen.
Eine Verlängerung der Buchung eines Raumes ist ebenfalls von Vorteil, da es
durchaus realistisch ist das ein Benutzer oder eine Benutzergruppe länger
benötigt, als die eigentliche Buchung im System hinterlegt ist.
Die identifizierten Risiken \bzw Änderungen sind zwar wichtig für einen
funktionierenden Ablauf des Systems, allerdings haben wir sie nicht als
zusätzliche \textit{Focal Use Cases} definiert. Um diese Abhängigkeiten
trotzdem  deutlich zu machen haben wir, wie bei den \textit{User Roles} auch, ein
Schaubild erstellt. Diese \textit{\citep[Kapitel~5]{softwareForUse}} befindet sich zur
besseren Lesbarkeit im \ref{[Anhang](use_case_map.png)}.

Wir haben dieses Feedback dafür genutzt die Anwendungsfälle zu überarbeiten
und neue Anwendungsfälle in Form der bereits bekannten \textit{Essential Use Cases}
zu erstellen. Die Aktualisierungen der Anwendungsfälle haben wir im
\ref{[Anhang](tabelle_Use_Cases_v2.md)} dokumentiert.
Zusätzlich befindet sich im \ref{[Anhang](gesprächsprotokoll.txt)} die Dokumentation
der wesentlichen Ergebnisse der Befragung.


\subsection{Fazit - Benutzungsmodellierung}
\label{sec:Fazit_Benutzungsmodellierung}

Das Ergebnis der Aufgabenmodellierung für das System haben wir in
Zusammenarbeit mit einigen potenziellen Benutzern ermittelt.
Die befragten Personen haben uns Feedback über die Relevanz der ermittelten
Aufgaben gegeben, welche wir für den weiteren Entwicklungsprozess benötigen.
Die identifizierten und in Use Cases formulierten Aufgaben stellen für den
Benutzer die Möglichkeiten da, mit dem System zu interagieren. Dabei werden
vermutlich im fertigen System auch mehrere Aufgaben kombinierbar sein.
Die genaue Implementation der Aufgaben innerhalb der Gestaltungslösung wird in
einem späteren Abschnitt behandelt. Anhand der Aufgaben lassen sich Risiken
ableiten die das Projekt gefährden können. Diese Risiken haben wir im [Anhang](Risiken bei der Aufgabenbestimmung)
dokumentiert und zusätzlich angegeben wie wir mit ihnen umgehen.


\section{Anforderungen}
\label{sec:Anforderungen}

\subsection{Einleitung}
\label{sec:Anforderungen_Einleitung}
Anhand der \textit{Role Models} aus der Benutzermodellierung und den Use Cases
aus der Benutzungsmodellierung lassen sich die Erfordernisse und Erwartungen
der Benutzer gegenüber dem System beschreiben. Diese identifizierten
Erfordernisse werden von uns im nächsten Schritt dazu verwendet Anforderungen
für das System zu formulieren.

\subsection{Anforderungen an das System}
\label{sec:Anforderungen an das System}
Im ersten Konzept des Projektes wurden bereits erste Anforderungen formuliert.
Diese sind unterteilt in Funktionale und Non-Funktionale Anforderungen.
Da im Verlauf der Konzeption und Entwicklung einiges abgeändert werden musste,
wurden auch diese Anforderungen noch einmal überarbeitet. Zusätzlich haben wir
sie um einige wichtige, auf das neu zu entwickelnde System bezogenen Punkten
erweitert. Die vollständige Liste der identifizierten Anforderungen des Systems
und des Projektes befindet sich im \ref{[Anhang](tabelle_anforderungen.md)}.


\section{Content Modelling}
\label{sec:Content_Modelling}

\subsection{Einleitung}
\label{sec:Content_Modelling_Einleitung}

Mit Hilfe der Benutzer- und Benutzungsmodellierung kann durch verschiedene
Methoden der "Interface Contents and Navigation" Thematik ein übersichtlicher
abstrakter Prototyp erstellt werden. Ergebnis dieser Vorgehensweise ist
eine "Content List" und eine "Content Navigation Map" mit deren Hilfe die
wichtigen Merkmale und Funktionen sowie die Übergänge zwischen den einzelnen
Aktionen und Dialogen innerhalb der Benutzerschnittstelle dargestellt werden
können.  
Wichtig hierbei ist, das es sich nur um einen abstrakten Prototypen handelt, also
das die Ergebnisse so wenig Designmerkmale und Lösungsvorschläge beinhalten wie
möglich. Weiterhin ist zu beachten, das genannte Eigenschaften wie ein
Drop-down Menü" nur ein Beispiel sind und für die fertige Gestaltungslösung noch
abgeändert werden können.

\subsection{Interface Content Modelling}
\label{sec:Interface_Content_Modelling}

Sinn der Content List ist es eine übersichtliche Auflistung der verschiedenen
"interaction spaces" mit ihren zugehörigen Inhalten zu erstellen.
Ein "interaction space" wird dabei in die verschiedenen Dialoge aufgeteilt.
Im Falle dieses Projektes werden die Interaction spaces \textit{userSpace},
\textit{instituteSpace} and \textit{adminSpace} mit ihren Dialogen
untergliedert ([siehe Content List]()).  
Der \textit{userSpace} besteht zum Beispiel aus den folgenden Dialogen:

\begin{itemize}
	\item Start Dialogue
	\item Filter Dialogue
	\item Reservation Dialogue
	\item Multi Reservation Dialogue
	\item Booking Dialogue
	\item Multi Booking Dialogue
\end{itemize}

Der erste Teil der Dialogbezeichnung wurde so gewählt das durch diesen
verdeutlicht wird, welche Aktion diesem Dialog vorausgegangen ist. Der Dialog
\textbf{Reservation Dialogue} beschreibt zum Beispiel den Kontext den der Benutzer
sieht, nachdem er erfolgreich einen Raum reserviert hat.
Die Dialoge werden weiterhin durch benötigte Inhalte erweitert und untergliedert.
Dabei wurde in diesem Projekt darauf geachtet, dass so wenig endgültige
Designeigenschaften der Elemente vorgegeben werden.

\subsection{Context Navigation Map}
\label{sec:Context_Navigation_Map}

Als Erweiterung zur Content List wurde eine \ref{[Context Navigation Map erstellt](context_navigation_map.png)}.
Diese verdeutlicht die Übergänge zwischen den einzelnen Dialogen. In diese
Context Navigation Map wurden Vorschläge für verschiedene
Interaktionsmöglichkeiten eingetragen. So kann die Beschriftung \textbf{[Buchen]} an
einem Übergang im späteren Projekt durch einen Button mit der Beschriftung
"Buchen" umgesetzt werden. Bei diesen Beschriftungen handelt es sich allerdings
nur um einen Vorschlag und keinen fertigen Entwurf, d.h. die Interaktion \bzw
der Übergang kann nach dem Interfacedesign auch durch eine andere Aktion
aktiviert werden.


\subsection{Fazit - Content Modelling}
\label{sec:Fazit_Content_Modelling}

Die aus dem Content Modelling entstandenen Artefakte können für das
Interfacedesign verwendet werden, da durch diese die wichtigen Informationen
und Übergänge zwischen verschiedenen Dialogen aufgezeigt werden. Die Artefakte
wurden allerdings so wenig vom Endgültigen Design vorgegeben wurde.


\subsection{User Interface Gestaltung}
\label{sec:User_Interface_Gestaltung}

Für das Design des User Interfaces, kurz UI, haben wir uns in diesem Projekt an
den fünf Regeln und sechs Prinzipien der Nutzbarkeit \citep{softwareForUse}
orientiert. Daraus ist als erste UI-Lösung ein Interface entstanden welches
verschiedene Aufgaben und Funktionalitäten gebündelt in Oberkategorien zur
Verfügung stellt. Damit ein Benutzer \zB einen Raum buchen kann muss er den
folgenden Interaktionspfad durchlaufen.

    Hauptmenü > Raum suchen > Raum / Räume suchen > Filter > Reservierung > Buchung

Um den Benutzer innerhalb dieses Interaktionspfades zu leiten haben wir uns
für den Stil eines Natural Language Interfaces entschieden, welcher einen
Benutzer mit Hilfe von ganzen Sätzen durch eine Aufgabe führt.
So entsteht \zB für einen Benutzer der erfolgreich einen Raum für sich
reserviert hat das folgende Interface:

\begin{figure}
	\centering
	%\includegraphics[scale=0.5]{reservierung.png}
	\caption{UI Dialog - Reservierung}
\end{figure}

Dieser Aufbau ermöglicht es erfahrenen Benutzer die wichtigen Informationen auf
einen Blick zu erfassen, hilft aber unerfahrenen Benutzern dennoch bei der
Durchführung ihrer Aufgabe.

Abschließend haben wir das entstandene UI selbstkritisch anhand der Regen und
Prinzipien bewertet und die Mängel aufgeschrieben.
Eines dieser Mängel war die nicht gegebene Selbstbeschreibungsfähigkeit also
das ein Benutzer ohne Erfahrungen mit unserem System nicht weiß was er zu
drücken hat um z.B. mehrere Räume zu reservieren und anschließend zu buchen.
Das UI wurde hinsichtlich der von uns festgestellten Mängel verbessert.
In dem aus dieser Iteration entstandenen UI findet der Benutzer alle Verfügbaren
Funktionen unseres Systems direkt auf der Startseite, sodass er die Lösung zu
seiner Aufgabe sofort finden kann. Dadurch hat sich der Interaktionspfad für
die Buchung eines Raumes folgendermaßen verändert.

    Hauptmenü > Einzelnen Raum suchen > Filter > Reservierung > Buchung

Außerdem wurden Aufgaben wie Einzelne und Mehrere Räume suchen von einander
getrennt was die Suche nach diesen Funktionen vereinfacht.

Das aus der Überarbeitung der Mängel entstandene UI wurde dann von Benutzern
ebenfalls im Hinblick auf die fünf Regeln und sechs Prinzipien ([s.o.]())
evaluiert. Die aus dieser Evaluation entstandenen Funktionswünsche und
Anmerkungen haben wir dann als Basis genommen um das existierende UI zu
erweitern. Das daraus entstandene UI verbessert die Übersicht einzelner
Interaktionsschritte in Bezug auf das Reservieren und Buchen von Mehreren
Räumen. So besitzt ein Benutzer der Mehrere Räume sucht nun die Möglichkeit
die Filter für jeden Raum einzustellen und kann anschließend die Reservierung
der Räume einzeln und getrennt von einander durchführen. Ebenso kann ein
Benutzer welcher bereits Mehrere Räume gebucht hat diese unabhängig voneinander
Stornieren und somit wieder im System freigeben.

