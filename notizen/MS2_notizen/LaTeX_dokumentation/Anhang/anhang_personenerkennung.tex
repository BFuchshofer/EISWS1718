% !TEX root = projektdokumentation.tex

\subsection{Druckplatten im Boden}
\label{anhang:Druckplatten_im_Boden}

Bei der Zählung von Personen durch eine druckempfindliche Matte im Eingangs-
Bereich kann durch Algorithmen sowohl die Anzahl der Personen als auch die
Laufrichtung dieser durch Analyse der Drucksensoren bestimmt werden.
Diese Matten können im Eingangsbereich zu den Räumen unter Fußmatten versteckt
werden und sind robust genug so dass keine Schäden durch Equipment welches
über diese gerollt wird entstehen sollte. Allerdings ist auf der Webseite des Anbieters
\citep{http://www.instacounting.com/intro.html} auf den wir uns bei dieser
Methode beziehen keiner Information dazu zu finden ob ein Gegenstand, zum
Beispiel ein Whiteboard mit rollen, beim Schieben über die Matte als Person
erkannt wird oder nicht. Hierfür müsste sofern möglich ein eigenes System
entwickelt oder das bestehende um Algorithmen zur Erkennung von bestimmten
Merkmalen, wie z.B. des Reifenabstandes, und der Zuordnung dieser  erweitert
werden. Weiterhin kann es durch das Verrutschen der Matte dazu kommen das Türen
nicht mehr richtig schließen. Dies kann durch das Befestigen der Matte auf dem
Boden behoben werden. Außerdem muss die Matte eine zu ermittelnde Mindestgröße
besitzen, damit Menschen mit größerer Schrittweite auch von dieser erkannt
werden.

\subsection{Lichtschranken und Bewegungsmelder}
\label{anhang:Lichtschranken_und_Bewegungsmelder}

Einige Methoden können zu einer Gruppe zusammengeschlossen werden, da diese
in etwa ähnlich funktionieren oder ähnliche Vor- und Nachteile besitzen. So
bilden die Bewegungsmelder und Lichtschranken zum Beispiel eine solche Gruppe.
Die Vorteile dieser Methoden sind zum Einen die Datenschutz unbedenkliche
Verwendung, da durch diese Methoden weder Bild- noch Tonmaterial aufgenommen
wird und somit auch keine Personenbezogenen Daten erhoben werden können.
Der Große Nachteil dieser Methoden ist, dass durch die Verwendung eines
einzelnen Gerätes weder die Laufrichtung noch die genaue Anzahl der Personen
bestimmen kann. So kann es zum Beispiel dazu führen, dass mehrere Personen
parallel oder leicht versetzt einen Raum betreten, vom System aber nur als
eine Person erkannt werden, da die Sensoren nur einen durchgängigen Impuls
senden. Des weiteren ist der große Abdeckungsbereich des Bewegungsmelders
eher hinderlich, da es dadurch auch zu Fehldeutungen kommen kann, wenn
ein Benutzer sich der Tür nur nähert, ohne hindurchgehen zu wollen.
Dieses Problem könnte für den Bewegungsmelder gelöst werden, indem künstlich
sein Wirkungsbereich eingedämmt wird. Außerdem könnte durch die Verwendung
mehrerer Geräte sowohl eine genauere Bestimmung der Anzahl sowie der
Laufrichtung von Personen bestimmt werden. Dies trifft nicht zu, wenn
mehrere Personen gleichzeitig mit Unterschiedlichen Laufrichtungen hindurch
laufen. Ein weiteres Problem tritt auf, wenn Gegenstände durch den
Wirkungsbereich geschoben werden, da der Sensor nicht zwischen Mensch und
Gegenstand unterscheiden kann. Um das Problem zu beseitigen könnte als
die Dauer des Impulses gemessen werden und anhand dessen entscheidet
das System ob es sich um einen Gegenstand oder eine Person handelt.

\subsection{Zählmechanik}
\label{anhang:Zählmechanik}

Die Methode eines Zählers oder einer Mechanik im Eingangsbereich oder der
Tür haben wir als einzelne Methode ausgeschlossen, da zwar eine effektive
Erkennung einer Interaktion mit einem Raum erkannt werden kann, jedoch
nicht festgestellt werden kann, ob der Raum betreten oder verlassen wird,
oder ob nur die Tür geöffnet wird. Es könnte durch die Kombination mit
einer anderen eine effektivere Methode generiert werden. Weiterhin könnte
durch Verwendung von Algorithmen bestimmt werden, ob eine Tür zum Beispiel
mehrfach aufgehalten wird und anhand der Zeit die eine Tür offen ist, wie
viele Personen möglicherweise hindurchgegangen sind. Dies führt allerdings
zu keiner genauen Anzahl an Personen im Raum.

\subsection{Wärmebildkameras}
\label{anhang:Wärmebildkameras}

Eine effektive Methode welche auch die persönlichen Daten der Benutzer
schützen würde wäre die Verwendung einer Wärmebildkamera mit zugehörigem
Algorithmus zur Personenzählung. Diese erreicht abhängig vom Algorithmus
eine sehr hohe Genauigkeit und ermöglicht wie zuvor gesagt die anonyme
Zählung der Personen in einem Raum. Wir haben diese Methode allerdings in
diesem Projekt ausgeschlossen, da durch die hohen Anschaffungskosten
eine Verwendung im Nutzungskontext einer Lehreinrichtung mit mehreren
hundert Räumen zum momentanen Zeitpunkt nicht denkbar ist.
Durch die Weiterentwicklung dieser Technologie sollte die Verwendung,
falls die Anschaffungskosten wesentlich geringer ausfallen, noch einmal
im Rahmen dieses Projektes evaluiert werden.

\subsection{Video-/Bildanalyse}
\label{anhang:Video_Bildanalyse}


Eine ähnliche und kostengünstigere Alternative bietet eine Analyse von
Video/- \bzw Bildmaterial mit der durch algorithmische Personenerkennung 
die Anzahl der Personen ermittelt werden kann. 
Der wichtigste Unterschied zur Wärmebildkamera ist hierbei
allerdings die fehlende Anonymität der Benutzer, da ein Video aus dem
inneren des Raumes aufgenommen wird. Es kann ebenfalls durch Verwendung
der richtigen Algorithmen eine hohe Genauigkeit erzielt, und so
eine effektive Personenzählung durchgeführt werden. Da die Analyse
solcher Videos sehr viel Rechenleistung benötigt und bei größeren Räumen
die Videos ebenfalls hochauflösender sein müssen könnte in diesem Projekt
eine Analyse von einzelnen Bildern, welche in festgelegten Intervallen
aufgenommen werden, durchgeführt werden. Dadurch könnte eine gute
Schätzung der Personen die sich in einem Raum befinden erreicht werden.
Die Frage des Datenschutzes müsste bei der Umsetzung des Projektes mit
dem Datenschutzbeauftragten der jeweiligen Lehreinrichtung besprochen
werden. Allerdings sollte darauf geachtet werden dass die Bilder nur
Lokal analysiert werden, danach gelöscht und nur das Ergebnis also
zum Beispiel '2 Personen' weiterverwendet wird.

\subsection{NFC}
\label{anhang:NFC}

Eine weitere Möglichkeit der Benutzererkennung innerhalb von Räumen
wären der Einsatz von NFC (Near Field Communication). 
Fast jedes neue Smartphone besitzt die Möglichkeit NFC einzusetzen.
Bekannt ist diese Methode z.B. durch das kontaktlose Bezahlen mit einem Smartphone. 
Die Reichweite von NFC liegt aktuell nur bei wenigen Zentimetern
\cite{https://developer.android.com/guide/topics/connectivity/nfc/index.html},
weswegen ein passives Erkennen von Benutzern nur schwer möglich ist.
Der Benutzer müsste, um über NFC erkannt zu werden, sein Endgerät sehr nahe an
das Lesegerät halten, was eine aktive Interaktion des Benutzers voraussetzt.
In unserem Projektkontext halten wir diese Methode für kontraproduktiv, da der
Benutzer sich aktiv am Scanner verifizieren muss was eine zusätzliche Aktivität bedeutet.

\subsection{BLE Beacons}
\label{anhang:BLE_Beacons}

Der Einsatz von Beacons wurde bereits bei der Standortbestimmung von Benutzern
angesprochen. Ebenfalls anwendbar wäre diese Methode bei der Bestimmung von
Benutzern innerhalb von Räumen.
Es wird also nach dem Standort innerhalb eines bestimmten Bereiches gesucht.
Dabei müsste das Endgerät des Benutzers als sendender Beacon umfunktioniert
werden und ein Empfangsgerät scannt innerhalb eines Raumes nach Bluetooth
Signalen. Sofern sich innerhalb des Raumes bzw. in einem bestimmten Umkreis
des Empfängers Bluetooth Signale empfangen lassen, kann das System davon
ausgehen das sich eine oder mehrere Personen im Raum befinden. Sobald das
Signal des Benutzers eine bestimmte Entfernung überschreitet, weiß das
System das dieser Benutzer den Raum vermutlich verlassen hat. Falls das Signal
des Benutzers wegen Störungen im Frequenzbereich vorübergehend nicht erreicht
werden kann, sollte das System den Benutzer nicht sofort als nicht mehr
vorhanden , \bzw den Raum als leer kennzeichnen, sondern eine gewisse
Zeitspanne einräumen in der weiter nach Benutzersignalen innerhalb eines Raumes
gescannt wird. Ist nach Ablauf dieser Zeit kein Benutzer erkannt worden, wird
der Raum wieder im System für andere Benutzer freigegeben. Der offensichtliche
Nachteil dieser Technologie für die Benutzererkennung ist, dass nicht jeder
Benutzer zwingend ein empfangsfähiges Endgerät dabei haben muss. Da immer nur
ein Benutzer einen Raum für \zB eine Gruppe von Personen bucht, besteht die
Gefahr das die Anzahl der Personen im Raum nicht erkannt werden kann. Verlässt
der Benutzer der den Raum gebucht hat den Raum wird er vom System nicht mehr
erkannt und der Raum wird wieder freigegeben obwohl sich noch arbeitende
Personen im Raum aufhalten. Ein zusätzliches Problem besteht im
gelegentlichen verlassen des Raumes, \zB für den Toilettenbesuch. Wird die im
System hinterlegte Zeit überschritten, wird der Raum wieder freigegeben obwohl
das vom Benutzer eigentlich nicht gewollt ist.


