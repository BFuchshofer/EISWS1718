% !TEX root = projektdokumentation.tex

\begin{table}[h]
	\caption{\textbf{Benutzer}}
 	\begin{tabularx}{\textwidth}{|c|c|X|c|c|}
		\rowcolor{heading}\textbf{Ressource} & \textbf{Methode} & \textbf{Semantik} & \textbf{content-type (req)} & \textbf{content-type (res)}\\ \hline
		/freeRoom & GET & gibt einen Raum zurück der in Abhängigkeit der Informationen im Body ausgewählt wurde. & application/json & application/json\\
		\rowcolor{odd}	/room/\{roomID\} & GET & gibt den aktuellen Status eines Raumes aus & - & application/json\\
		/room/\{roomID\}/status & PUT & ändert den Status des angegebenen Raumes in den angegebenen Status & application/json & application/json\\
 	\end{tabularx}
 \end{table}
 
 \begin{table}[h]
	\caption{\textbf{Administrator}}
 	\begin{tabularx}{\textwidth}{|c|c|X|c|c|}
		\rowcolor{heading}\textbf{Ressource} & \textbf{Methode} & \textbf{Semantik} & \textbf{content-type (req)} & \textbf{content-type (res)}\\
						/room & PUT & gibt einen Raum zurück der in Abhängigkeit der Informationen im Body ausgewählt wurde. & application/json & application/json\\
		\rowcolor{odd}	/room/\{roomID\} & PUT & gibt den aktuellen Status eines Raumes aus & - & application/json\\
						/room/\{roomID\} & DELETE & entfernt einen Raum aus dem System & - & -\\
		\rowcolor{odd}	/room/\{roomID\} & GET & enthält alle Informationen die zu einem Raum gespeichert wurden & - & application/json\\
 	\end{tabularx}
 \end{table}