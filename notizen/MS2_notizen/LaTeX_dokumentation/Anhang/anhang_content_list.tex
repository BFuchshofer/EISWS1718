% !TEX root = projektdokumentation.tex

\begin{itemize}[noitemsep]
	\item \textbf{userSpace}
	\begin{itemize}
		\item \textbf{Start Dialogue}
		\begin{itemize}
			\item \textit{toolSelection}\\Eine Auswahl an verfügbaren Funktionen und Tools.
		\end{itemize}
		\item \textbf{Filter Dialogue}
		\begin{itemize}
			\item \textit{roomEquipment}\\Eine Auswahl an möglichem Equipment.
			\item \textit{roomProperties}\\Eine Auswahl an möglichen Raumeigenschaften.
			\item \textit{roomSize}\\Auswahl der Größe des Raumes.
		\end{itemize}
		\item \textbf{Reservation Dialogue}
		\begin{itemize}
			\item \textit{roomNumber}\\Anzeige der Identifikation des reservierten Raumes.
			\item \textit{reservationDuration}\\Anzeige der restlichen Dauer der Reservierung.
		\end{itemize}
		\item \textbf{Multi Reservation Dialogue}
		\begin{itemize}
			\item \textit{roomNumberList}\\Anzeige der Identifikationen der reservierten Räume.
			\item \textit{reservationDuration}\\Anzeige der restlichen Dauer der Reservierungen.
		\end{itemize}
		\item \textbf{Booking Dialogue}
		\begin{itemize}
			\item \textit{roomNumber}\\Anzeige der Identifikation des gebuchten Raumes.
			\item \textit{bookingDuration}\\Anzeige der restlichen Dauer der Buchung.
		\end{itemize}
		\item \textbf{Multi Booking Dialogue}
		\begin{itemize}
			\item \textit{roomNumberList}\\Anzeige der Identifikationen der gebuchten Räume.
			\item \textit{bookingDuration}\\Anzeige der restlichen Dauer der Reservierungen.
		\end{itemize}
	\end{itemize}
	
	\item \textbf{instituteSpace}
	\begin{itemize}
		\item \textbf{Start Dialogue}
		\begin{itemize}
			\item \textit{toolSelection}\\Eine Auswahl an verfügbaren Funktionen und Tools.
		\end{itemize}
		\item \textbf{Search Dialogue}
		\item \textbf{Block Dialogue}
		\begin{itemize}
			\item \textit{roomNumber}\\Anzeige der Identifikation des geblockten Raumes.
			\item \textit{blockingDuration}\\Anzeige der restlichen Dauer der Blockung.
		\end{itemize}
	\end{itemize}
	
	\item \textbf{adminSpace}
	\begin{itemize}
		\item \textbf{Start Dialogue}
		\begin{itemize}
			\item \textit{toolSelection}\\Eine Auswahl der nur für den Administrator verfügbaren Funktionen und Tools.
		\end{itemize}
		\item \textbf{Search Dialogue}
		\item \textbf{Add Room Dialogue}
		\begin{itemize}
			\item \textit{roomInformationForm}\\Eine Formular mit notwendigen und optionalen Raumeigenschaften und Rauminformationen.
		\end{itemize}
		\item \textbf{Edit Room Dialogue}
		\begin{itemize}
			\item \textit{roomInformationForm}\\Ein Formular mit allen im System verfügbaren Rauminformationen.
		\end{itemize}
		\item \textbf{Delete Room Dialogue}
		\begin{itemize}
			\item \textit{roomNumber}\\Anzeige der Identifikation des zu löschenden Raumes
		\end{itemize}
		\item \textbf{Add Equipment Dialogue}
		\begin{itemize}
			\item \textit{equipInformationForm}\\Ein Formular mit notwendigen und optionalen Equipmenteigenschaften und Equipmentinformationen.
		\end{itemize}
		\item \textbf{Edit Equipment Dialogue}
		\begin{itemize}
			\item \textit{equipInformationForm}\\Ein Formular mit allen im System verfügbaren Equipmentinformationen.
		\end{itemize}
		\item \textbf{Delete Equipment Dialogue}
		\begin{itemize}
			\item \textit{equipID}\\Anzeige der Identifikation des zu löschenden Equipments.
		\end{itemize}
		\item \textbf{Add Eventplan Dialogue}
		\begin{itemize}
			\item \textit{eventplanInformationForm}\\Ein Formular mit notwendigen und optionalen Eventplaninformationen.
		\end{itemize}
		\item \textbf{Edit Eventplan Dialogue}
		\begin{itemize}
			\item \textit{eventplanInformationForm}\\Ein Formular mit allen im System verfügbaren Eventplaninformationen.
		\end{itemize}
		\item \textbf{Delete Eventplan Dialogue}
		\begin{itemize}
			\item \textit{eventplanInformation}\\Anzeige der Identifikation des zu löschenden Eventplans.
		\end{itemize}
		\item \textbf{Add Knoten Dialogue}
		\begin{itemize}
			\item \textit{knotenInformationForm}\\Ein Formular mit notwendigen und optionalen Knoteninformationen.
		\end{itemize}
		\item \textbf{Edit Knoten Dialogue}
		\begin{itemize}
			\item \textit{knotenInformationForm}\\Ein Formular mit allen im System verfügbaren Knoteninformationen.
		\end{itemize}
		\item \textbf{Delete Knoten Dialogue}
		\begin{itemize}
			\item \textit{nodeID}\\Anzeige der Identifikation des zu löschenden Knoten.
		\end{itemize}
	\end{itemize}
\end{itemize}

\subsubsection*{Erklärung der Dialoge}
    \paragraph*{Start Dialogue}
        Dieser Dialog soll für alle interaction spaces den Kontext beschreiben den der
        Benutzer aufruft wenn er auf das System zugreift.
        Von ihm aus kann auf die wichtigen Funktionen des Systems zugegriffen werden.
        Für den instituteSpace und userSpace wurde dieser Dialog zusammen gelegt,
        da diese einen ähnlichen Anwendungsbereich beinhalten.

   \paragraph*{Search Dialogue}
        Der Search Dialogue des instituteSpaces stellt den Kontext der Suche nach einem
        bestimmtem Raum dar, d.h. die Suche mit Hilfe einer bekannten Raumnummer oder
        Raumeigenschaften.
        Der Search Dialogue des adminSpaces stellt den gleichen Kontext dar, erweitert
        um die Veranstaltungen, das Equipment und die Knotenpunkte mit der Bedingung
        dass die Identifikation dieser bekannt ist.

    \paragraph*{Reservation Dialogue / Multi Reservation Dialogue}
        Diese Dialoge stellen den Kontext dar in dem sich ein Benutzer befindet, wenn
        er einen Raum reserviert hat, also dessen Raumnummer erhalten hat und sich auf
        dem Weg zu diesem befindet. Die Variante mit Multi ist lediglich der Kontext
        in dem sich ein Benutzer befindet wenn er mehrere Räume gleichzeitig reserviert
        hat, sofern er die benötigte Berechtigung besitzt.

    \paragraph*{Booking Dialogue / Multi Booking Dialogue}
        Ähnlich wie die o.g. Reservierungs Dialoge beziehen sich diese auf den Kontext
        nach dem ein Nutzer einen Raum gebucht bzw. belegt hat. Die Variante mit Multi
        ist lediglich erneut der Kontext mit mehreren gebuchten bzw. belegten Räumen.

    \paragraph*{Add * / Edit * / Delete *}
        Diese Dialoge umfassen die Kontexte die durch den Admin hervorgerufen werden
        um eine Ressource aus dem Systemzu verändern , hinzuzufügen oder zu entfernen.
