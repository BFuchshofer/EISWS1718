% !TEX root = ../projektdokumentation.tex

\section*{Vorwort}
\label{sec:Vorwort}

Die folgende Projektdokumentation wurde im Rahmen des Moduls
"Entwicklungsprojekt interaktive Systeme" erstellt und befasst sich mit dem
Thema der effektiven
Raumfindung innerhalb einer Lehreinrichtung und der Planung eines Systems,
welches diese Aufgabe unterstützt und für den Benutzer in angemessener Zeit
erfüllt.
Da das Konzept, welches diesem Projekt vorausgegangen ist, einige Mängel
aufgewiesen hat und sich in Teilen nicht für die Umsetzung in einem Projekt
durchführen lies, haben wir uns im voraus dazu entschlossen dieses zu
überarbeiten.
Das Nutzungsproblem aus unserem Konzept wurde durch die Berücksichtigung eines
Zeitfaktors erweitert. Daraus resultierend wurde auch die Anwendungsdomäne und
ein Großteil des Nutzungskontextes überarbeitet.
Aus der von uns für das Konzept entworfenen Gestaltungslösung und dem
zugehörigen Prototypen wurde der Großteil stark überarbeitet und durch eine neu
spezifizierte Gestaltungslösung erweitert.
Ziel der Überarbeitung war es aus dem einfachen Buchungssystem des
Konzeptentwurfs ein komplexeres Raumbelegungssystem zu entwickeln, welches die
automatische Erfassung und Auswertung von Daten ermöglicht.


\section{Einleitung}
\label{sec:Einleitung}

Der Faktor Zeit spielt in vielen Lebenslagen eine entscheidende Rolle! Sei es
die Einhaltung von Fristen für Abgaben von Dokumentationen, Präsentationen oder
entwickelten Produkten, die genaue Dokumentation von geleisteten Arbeitsstunden
oder der Wegstrecke die zurückgelegt werden muss, um von Punkt A nach Punkt B
zu gelangen. In vielen Bereichen ist es wichtig Zeit einzusparen, da Zeit
bekanntermaßen Geld ist. In nicht erwerbstätigen Bereichen ist der Faktor Zeit
dahingehend wichtig, da mit mehr Zeit auch mehr Aktivitäten ausgeführt werden
können. Es liegt also nahe, auch in diesen Bereichen die verfügbare Zeit zu
optimieren. Ein Punkt an dem dieses Projekt anknüpfen soll, ist das schnelle
und unkomplizierte finden von Räumen die den Erfordernissen und Erwartungen
einer Person oder Personengruppe entsprechen.


\subsection{Nutzungsproblem}
\label{sec:Nutzungsproblem}

Um die Belegungen von Räumen zu verwalten, benutzen die meisten Unternehmen und
Organisationen den klassischen Belegungsplan der sich vor jedem Raum befindet,
indem sich Mitarbeiter oder andere Personen über die Belegung von bestimmten
Räumen informieren können. Dieser Belegungsplan ist meist eine auf Papier
gedruckte Tabelle mit den Tagen und Uhrzeiten wann dieser Raum von wem
offiziell belegt ist. Daraus lässt sich dann ableiten, wann ein Raum nicht
belegt, und theoretisch von anderen Personen genutzt werden kann. Allerdings
sind in diesen Plänen in der Regel nur fest definierte Belegungen, die oft
wöchentlich wiederkehren, verzeichnet. Dynamische Nutzungen eines Raumes lassen
sich daraus meist nicht ableiten. Die logischste Möglichkeit herauszufinden, ob
ein Raum leer ist, ist natürlich diesen einfach zu öffnen und nachzuschauen.
Dies setzt allerdings voraus, dass man sich bereits vor solch einem
entsprechenden Raum befindet. Um auch Personen die Raumsuche zu ermöglichen,
die sich nicht zufällig in einem Gang mit möglichen Arbeitsräumen befinden,
oder vor einem entsprechenden Raum stehen, wird ein System benötigt, dass über
Entfernung überprüft, ob ein Raum zur Verfügung steht oder nicht. Um den Faktor
Zeit ins Spiel zu nehmen, wird ein System benötigt, dass in Abhängigkeit der
aktuellen Position des Benutzers einen freien Raum angibt, der sich in der Nähe
des Benutzers befindet. Muss man sich erst in ein Gebäude, Stockwerk oder Gang
begeben um Informationen über den Status eines Raumes einzuholen, wird Zeit in
Form von zusätzlichen Laufwegen verschwendet, da die Möglichkeit besteht, dass
kein Raum in diesem Gebäude aktuell zur Verfügung steht und man so andere
Bereiche aufsuchen muss. Neben einer Zeitersparnis eliminiert man außerdem den
Störfaktor der entsteht wenn eine nach einen freien Raum suchende Person einen
Raum betritt in dem bereits gearbeitet wird. Die Möglichkeit bei der Raumsuche
Zeit zu sparen, die man somit für wichtige Arbeiten verwenden kann, ist also
durchaus relevant um effektiver arbeiten zu können.   
In diesem Projekt soll es um die Einsparung von Zeit in Form von verkürzten
Laufwegen bzw. einer verkürzten Suchzeit für einen Raum gehen.


\subsection{Technologie unabhängiger Lösungsvorschlag}
\label{sec:Technologie_unabhängiger_Lösungsvorschlag}

Um dem Problem der spontanen und unstrukturierten Raumbelegung entgegen zu
wirken ist ein System nötig, welches sowohl die verfügbaren Räume, als auch
wöchentlich wiederkehrende und spontane Belegungen aufnimmt, um damit einer
Person oder Personengruppe auf Raumsuche einen Raum vorzuschlagen. Dieser
Raumvorschlag sollte in Abhängigkeit der aktuellen Position des Benutzers
erfolgen, um einen kurzen Laufweg gewährleisten zu können. Das System muss den
Standort des Benutzers automatisch ermitteln und die Berechnung des
Raumvorschlages einbeziehen können. 
Der Benutzer sollte dabei die Möglichkeit haben sich aus der Ferne über den
Status eines Raumes informieren zu können.
Bei Bedarf sollte die Möglichkeit bestehen diesen Raum für eine bestimmte
Zeitspanne zu buchen, damit gewährleistet wird das kein anderer Benutzer diesen
Raum für seine Zwecke verwendet.
Dabei sollte darauf geachtet werden, dass der vorgeschlagene Raum auch dem
Vorhaben der Person genügt. Eventuelle Bedürfnisse über die Rauminhalte müssen
dem System vom Benutzer mitgeteilt werden.


\section{Ziele}
\label{sec:Ziele}

\subsubsection{Einleitung}
\label{sec:Ziele_Einleitung}

Anhand des identifizierten Nutzungsproblems und des ermittelten
Alleinstellungsmerkmals, lassen sich konkrete Ziele aufstellen die so ein
System  erfüllen muss um einen qualitativen Lösungsansatz zu erzielen.   
Die folgenden Ziele sind unterteilt in strategische, taktische und operative
Ziele, die zu Teilen während des Entwicklungsprozesses gebildet wurden.

\subsubsection{Strategische Ziele}
\label{sec:Strategische_Ziele}

\begin{itemize}
	\item Das fertige System in unserem Projekt \textbf{muss} es dem Benutzer
	ermöglichen innerhalb von 30 Sekunden einen freien Raum zu erhalten der
	den Erfordernissen des Benutzers entspricht.
	\item Dabei \textbf{muss} das System alle zugänglichen Räume innerhalb
	der Lehreinrichtung in Echtzeit adressieren und verwalten können.
	\item Außerdem \textbf{muss} das System Gegenstände die sich in einem Raum
	befinden und diesen charakterisieren erkennen, adressieren und verwalten
	können.
\end{itemize}

\subsubsection{Taktische Ziele}
\label{sec:Taktische_Ziele}

\begin{itemize}
	\item Um eine effektive Raumsuche gewährleisten zu können \textbf{muss} der
	Standort des Benutzers in die Raumauswahl mit einbezogen werden.
\end{itemize}

\subsubsection{Operative Ziele}
\label{sec:Operative_Ziele}

\begin{itemize}
	\item Der Benutzer \textbf{muss} eine Auswahl an benötigten Gegenständen
	treffen, die in die Raumsuche mit einbezogen werden.
	\item Der Standort des Benutzers \textbf{wird} automatisch bestimmt.
	\item Das Betreten/Verlassen eines Gegenstandes aus/in einen Raum \textbf{muss}
	vom System erkannt werden.
	\item Alle für den Raum relevanten Gegenstände \textbf{müssen} markiert werden.
	\item Vor Betreten eines Raumes \textbf{muss} festgestellt werden ob ein
	Benutzer eine Aufenthaltsberechtigung hat.
	\item Falls keine Aufenthaltsberechtigung vorliegt \textbf{muss} eine solche
	\ggfs vom System beantragt werden.
\end{itemize}


\subsection{Domänenspezifische Recherche}
\label{sec:Domänenspezifische_Recherche}

\subsubsection{Einleitung}
\label{sec:Domänen_Einleitung}

Das beschriebene Problem der dynamischen Raumsuche kann so gut wie in jeder
Organisation auftreten in der es mehrere Räume gibt, die für unterschiedliche
Arbeiten genutzt werden können. Vorallem in Organisationen die eine Vielzahl
von verschiedensten Räumen besitzen, möglicherweise sogar über mehrere Etagen
oder Komplexe verteilt, ist die Gefahr groß, dass eine arbeitsplatzsuchende
Person Zeit mit langen Laufwegen verschwendet, da ihr nicht bekannt ist wo sich
der nächste frei nutzbare Raum befindet. Das Problem nimmt größere Ausmaße an,
je mehr Personen einen freien Raum suchen.

\subsubsection{Unternehmen}
\label{sec:Unternehmen}

Dadurch das große Unternehmen in der Regel auch viele Mitarbeiter beschäftigen,
und diese Arbeitsmöglichkeiten benötigen, kommt dieser Nutzungsbereich als
Möglichkeit in Frage. Mitarbeiter brauchen ggf. Räume für Präsentationen mit
speziellem Equipment wie Beamer oder Whiteboards, oder benötigen einen Raum in
dem mit mehreren Personen kreativ gearbeitet werden kann. Genauso werden
Arbeitsmöglichkeiten benötigt, in dem einzelne Personen in Ruhe ihre Arbeit
erledigen können. In Unternehmen herrscht allerdings meistens eine
kontrollierte Arbeitsumgebung, wo über den Tag schon geplant wurde, welche
Personen oder Personengruppen was für Räume oder Equipment benötigen, da in
Projekten in der Regel ein konkreter Ablauf von Tätigkeiten vorliegt. Außerdem
werden Arbeitsmöglichkeiten für Mitarbeiter meist dadurch gewährleistet, dass
sie einen festen Schreibtisch oder Büro besitzen.

\subsubsection{Fazit - Unternehmen}
\label{sec:Fazit_Unternehmen}

In großen Unternehmen mit vielen Mitarbeitern besteht in der Theorie ein Bedarf
an einem System, dass ein besseres Raummanagement und Zeitersparnis liefert,
allerdings werden diese Faktoren durch die kontrollierte Arbeitsumgebung von
Unternehmen oft selbst gelöst.

\subsubsection{Lehreinrichtungen}
\label{sec:Lehreinrichtungen}

Lehreinrichtungen mit vielen Schülern, Studenten oder Auszubildenden Personen
sind ein guter Kandidat um eine Lösung für ein solches Nutzungsproblem zu
finden. Dadurch das Lerner neben festen Veranstaltungen noch beispielsweise
Gruppenarbeiten oder Stillarbeit erledigen, ergibt sich eine sehr variable
Anzahl an Personen, die im gleichen Zeitraum einen freien Raum benötigen.
Die Nutzergruppen beschränken sich also auf die Angestellten und Mitarbeiter
der Lehreinrichtung, sowie alle Lerner die interessiert an einer Funktion sind,
die es ihnen ermöglicht, schnell und einfach einen freien Raum zu finden, der
ihren Ansprüchen genügt.
Dabei ist nicht nur das finden von freien Räumen sondern, je nach Kontext, die
Information ob ein Raum gerade belegt ist interessant. Eventuell vorhandenes
Reinigungs- oder Wartungspersonal kann somit z.B. erfahren, welcher Raum
aktuell belegt ist um die darin befindlichen Personen nicht zu stören.
Desweiteren können Mitarbeiter die Wartungsarbeiten innerhalb eines Raumes
durchführen herrausfinden, wann sie für ihre Arbeiten den Raum belegen und
nutzen können. Die Laufwege und dadurch das Zeitmanagement von Mitarbeitern
kann somit ebenfalls optimiert werden.

\subsubsection{Fazit - Lehreinrichtungen}
\label{sec:Fazit_Lehreinrichtungen}

Die Anwendungsdomäne der Lehreinrichtungenn mit Fokus auf einer großen
Personen und Raumanzahl bietet durch die Vielzahl an Bedürfnissen der
einzelnen Benutzer und der benötigten hohen Produktivität der Arbeiten
eine gute und sinnvolle Möglichkeit.

\subsubsection{Fazit zur Domänen spezifischen Recherche}
\label{sec:Fazit_zur_Domänen_spezifischen_Recherche}

Im direkten Vergleich von Unternehmen und Lehreinrichtungen bietet die
Anwendungsdomäne der Lehreinrichtung mit vielen beteiligten Personen und vielen
Räumen das ideale Umfeld für ein System, dass den Benutzern dabei hilft Zeit
bei der Suche nach einem freien Raum zu sparen. Für Lehreinrichtungen mit
wenigen Räumen und/oder Benutzern, ist dieses System weniger relevant, da eine
wirkliche Zeitersparnis in den meisten Fällen nicht gewährleistet werden kann.


\subsection{Marktrecherche}
\label{sec:Marktrecherche}

\subsubsection{Einleitung}
\label{sec:Markt_recherheEinleitung}

Schaut man sich im Internet nach Angeboten zu Raumplanungssystemen um, finden
sich viele Angebote die in diese Richtung gehen. Im Zuge einer Marktrecherche
wurden verschiedene Anbieter von Raumplanungs-Managementsystemen auf ihre
angebotenen Funktionalitäten überprüft, Vor- und Nachteile ermittelt und auf
den benötigten Projektkontext hin verglichen.

\subsubsection{Locaboo}
\label{sec:Locaboo}

Locaboo \citep{locaboo} wirbt mit einem Belegungsplan für Sport-,
Bildungs- und Freizeitstätten. Dabei soll die Softwarelösung aus dem 
Hause LOY GmbH die Verwaltung von verschiedenen Räumen und Bereichen
übersichtlich und zeitsparend möglich machen. Genannt werden Funktionen die
eine effektivere Auslastung der angebotenen Räume oder Bereiche ermöglichen,
sowie die einfache Vermietung von Ressourcen. Das übersichtliche angebotene
Dashboard soll eine schnelle und zuverlässige Übersichtsseite der vorhandenen
Ressourcen, und der Einstellung von Informationen wie Raumdaten oder
Preisquellen ermöglichen.

\begin{itemize}
	\item Vorteile:
		\begin{itemize}
			\item funktionales und übersichtliches Dashboard
			\item effektives Raum- und Ressourcenmanagement
			\item Möglichkeit um für verschiedene Einsatzzwecke Spezialmodule zu buchen
			\item Möglichkeit Teilflächen von \bspw Räumen zu buchen.
			\item flexible Abrechnungsmethoden
			\item Einbindung von Kundenaktivitäten
		\end{itemize}
	\item Nachteile:
		\begin{itemize}
			\item Fokus auf Verfügbarkeit und Auslastung von Räumen und Ressourcen
			\item keine flexible Raumsuche in Abhängigkeit des Standortes des Benutzers
		\end{itemize}
\end{itemize}

\paragraph{Fazit - Locaboo}
\label{sec:Fazit_Locaboo}

Diese Softwarelösung ist als ein digitaler Kalender zu sehen der im Prinzip die
digitalisierte und standortunabhängige Version des Belegungsplanes vor jedem
Raum darstellt. Es gibt zwar die Möglichkeit schnell und einfach Belegungen
hinzuzufügen oder zu entfernen, jedoch liegt der Fokus mehr auf der Verwaltung
von Räumen aus Veranstaltersicht, und nicht aus der Sicht der einzelnen Raumnutzer.  

\subsubsection{INTIME}
\label{sec:INTIME}

INTIME \citep{intime} von \textit{COMTEC} bietet ebenfalls die Möglichkeit
Räume und Gebäude besser verwalten zu können. Dabei bietet es auch neben
zahlreichen Zusatzfunktionen die für Lehreinrichtungen weniger relevant sind,
für den einzelnen Benutzer die Möglichkeit schnell einen Raum zu finden der
seinen Erwartungen und Erfordernissen entspricht.

\begin{itemize}
	\item Vorteile:
		\begin{itemize}
			\item effektives Raum- und Ressourcenmanagement
			\item Einzelpersonen oder Gruppen können nach einem Raum suchen
			\item Möglichkeit bestimmte Gegenstände zu einem Raum dazu zu buchen
			\item Kalenderfunktion
			\item Bestandsaufnahmen von vorhandener Technik und Ressourcen
		\end{itemize}
	\item Nachteile:
		\begin{itemize}
			\item keine flexible Raumsuche in Abhängigkeit des Standortes des Benutzers möglich
		\end{itemize}
\end{itemize}

\paragraph{Fazit - INTIME}
\label{sec:Fazit_INTIME}

Die Softwarelösung \textit{INTIME} beschränkt sich auf das Verwalten und Managen von
Ressourcen. Die Benutzer des Systems sind Personen die auch aktiv einen Raum
suchen. Ob sich der Raum allerdings in der Nähe des Benutzers oder am anderen
Ende des Gebäudekomplexes befindet ist nicht ersichtlich. Hier wird der Faktor
Zeit also nicht vollständig berücksichtigt, der Fokus liegt vielmehr auf der
Verfügbarkeitsüberprüfung.

\subsubsection{Online-Raumverwaltung}
\label{sec:Online-Raumverwaltung}

Die Softwarelösung zur Raumverwaltung der Firma \textit{OMOC interactive}) liefert eine
cloudbasierte Möglichkeit um es dem Benutzer zu ermöglichen, Raumreservierungen
vorzunehmen. Die Software ist von verschiedenen Endgeräten aus bedienbar und
bietet alle gebuchten Funktionalitäten komfortabel an. \textit{OMOC interactive} wirbt
mit einer leichten Integration auf Websites, Outlook und auf Wunsch auch die
Übernahme von bereits bestehenden Daten in das neue System. Die Software bietet
eine Verwaltung von Veranstaltungen und Ressourcen ohne Installation und
Wartungskosten für den Kunden.

\begin{itemize}
	\item Vorteile:
		\begin{itemize}
			\item Verwaltung von komplexen Veranstaltungen
			\item Planen und abbrechen von Veranstaltungen
			\item von überall bedienbar
			\item Cloudbasiert
			\item Rechnung/Mahnungen mit einem Klick verschicken
			\item Integration der Software in verschiedene Systeme möglich
		\end{itemize}
	\item Nachteile:
		\begin{itemize}
			\item keine flexible Raumsuche in Abhängigkeit des Standortes des Benutzers.
		\end{itemize}
\end{itemize}

\paragraph{Fazit - Online-Raumverwaltung}
\label{sec:Fazit_Online-Raumverwaltung}

Die Funktionen der Online-Raumverwaltung \citep{omoc} der
\textit{OMOC interactive} bietet zwar eine praktische und cloudbasierte
Möglichkeit um Räume für Veranstaltungen zu organisieren und verwalten, hier
liegt der Fokus aber mehr auf der Vermietung solcher Räume. Der Endbenutzer,
mit seinem Interesse im kompletten Prozess der Raumsuche Zeit zu sparen, ist
hier nicht adressiert.   

\subsubsection{Raumplaner der TH Köln}
\label{sec:Raumplaner_der_TH_Köln}

Innerhalb der Anwendungsdomäne der Lehreinrichtungen stellt die 
Technische Hochschule Köln (Campus Gummersbach) ein Online-Tool \citep{thkoeln}
zur Verfügung, mit dem Einsicht in Raumbelegungen innerhalb des Campus
vorgenommen werden können. Es besteht die Möglichkeit Informationen zu
bestimmten Räumen einzuholen die unter anderem Belegungen, Zeitrahmen der
Belegung und Ausstattung des Raumes beinhalten.

\begin{itemize}
	\item Vorteile:
		\begin{itemize}
			\item Übersicht über alle verfügbaren Räume
			\item bei vielen Räumen Angaben zum Rauminhalt
			\item Start- und Endzeitpunkt der Belegung
		\end{itemize}
	\item Nachteile:
		\begin{itemize}
			\item nur wöchentlich wiederkehrende Veranstaltungen
			\item lediglich eine Digitalisierung der Aushangpläne vor jedem Raum
			\item keine flexible Raumsuche in Abhängigkeit des Standortes des Benutzers
		\end{itemize}
\end{itemize}

\paragraph{Fazit - Online-Tool der TH Köln}
\label{sec:Fazit_Online-Tool der TH Köln}

Das Online-Tool der TH Köln bietet zwar eine Übersicht über alle vorhandenen
Räume innerhalb der Lehreinrichtung und liefert in vielen Fällen auch Angaben
über das Equipment der Räume mit, allerdings sind auch hier keine
Zeiteinsparungen im kompletten Raumsuchungsprozess möglich. Die dynamische
Erfassung von Raumbelegungen ist auch hier nicht gewährleistet.


\subsubsection{Fazit zur Marktrecherche}
\label{sec:Fazit_zur Marktrecherche}

Es gibt viele Softwarelösungen für ein besseres Raummanagement, jedoch ist
keine davon darauf ausgelegt, dem einfachen Benutzer einen freien Raum in
seiner Nähe anzuzeigen und diesen für ihn zu reservieren / buchen. Fast alle
gefundenen Lösungen beziehen sich auf die Vermietung von Räumen oder
Veranstaltungsorten die von einem Unternehmen verwaltet werden.
Der Anwendungsbereich ist also vorrangig im kommerziellen Bereich. Benutzer
sind in den meisten Fällen die Unternehmen selbst, die eine Managementfunktion
benötigen. Dabei bieten aber fast alle Softwarelösungen auch die Möglichkeit
für einfache Endbenutzer Informationen über einen freien Raum zu bekommen,
den sie gerade benötigen. Diese Funktionen beschränken sich aber auf den
Verfügbarkeits-Check eines Raumes, unabhängig vom Standort des Raumes zur
Person. Den gefundenen Softwarelösungen fehlt somit ein wichtiges Feature um im
kompletten Prozess der Raumsuche, beginnend bei dem Verfügbarkeitscheck eines
Raumes, über den Laufweg zu diesem Raum, bis zur tatsächlichen Belegung.


\subsection{Alleinstellungsmerkmale}
\label{sec:Alleinstellungsmerkmale}

Betrachtet man die identifizierten Konkurenzprodukte, erkennt man das der
Verfügbarkeitscheck von bestimmten Räumen bei den meisten Produkkten eine
kleine, wenn auch nicht unbedeutende, Teilfunktion darstellt. Zeit wird nur
dahingehend eingespart, da man sich schnell und einfach über einen freien Raum
informieren kann. Die Möglichkeit Zeit zu sparen, indem man kürzere Laufwege
und Suchzeiten berechnet, ist in keinem der Produkte namentlich erwähnt.
Unser System wird also genau an diesem Punkt anknüpfen um die gefundenen Lücken
zu schließen. Konkret bedeutet das, das sich folgendes Alleinstellungsmerkmal
für unser System im Vergleich zur gefundenen Konkurenz ergibt:

\begin{itemize}
	\item Zeitersparnis durch insgesamt verkürzte Laufwege und Suchzeiten.
\end{itemize}


\subsection{Risiken}
\label{sec:Risiken}

\subsubsection{Einleitung}
\label{sec:Risiken_Einleitung}

Anhand der vorrangegangenen Recherche an Informationen zu einer möglichen
Realisierung des Systems wurden erste Risiken identifiziert. Im Verlauf des
Projektes wird diese Liste an Risiken bezüglich auf verwendete Technologien
und neuen Erkentnissen stetig aktualisiert. Da noch keine Technologien
feststehen die verwendet werden sollen, sind die Risiken aktuell nur auf das
Projekt als ganzes bezogen. Sobald Technologien feststehen, und sich darauf
neue Risiken ergeben, wird die Liste im Anhang vervollständigt.

\subsubsection{Projektrisiken}
\label{sec:Projektrisiken}

\begin{itemize}
	\item Benutzer eines Raumes blockieren den Raum körperlich über die im System hinterlegte Zeitspanne hinaus.
	\item Das System liefert einen freien Raum nicht in absehbarer Zeit.
	\item Das Finden eines freien Raumes dauert mit Hilfe der Anwendung länger als ohne.
	\item Es lässt sich nicht gewährleisten das ein Raum auch tatsächlich frei ist, wenn dem Benutzer
	dieser ausgegeben wird. ( andere Benutzer blockieren den Raum \zB weil er nicht abgeschlossen ist)
\end{itemize}