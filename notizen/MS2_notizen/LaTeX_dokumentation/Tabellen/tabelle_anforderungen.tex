% !TEX root = projektdokumentation.tex

\section{Anforderungen}

\subsection{Funktionale Anforderungen}
\label{sec:Funktionale_Anfoderungen}

\subsubsection{Selbstständige Systemaktivität}
\label{sec:Selbstständige_Systemaktivität}

\begin{itemize}
	\item Das System sollte dem Benutzer eine vordefinierte Auswahl an Raumspezifikationen zur Filterung zur Verfügung stellen.
	\item Das System muss Benutzereingaben verarbeiten können.
	\item Das System muss Benutzereingaben auswerten können.
	\item Das System muss anhand von ihm zur Verfügung stehenden Daten eine Raumauswahl treffen können.
	\item Das System muss gewährleisten das Lerner nur einen Raum gleichzeitig reservieren können.
	\item Das System muss gewährleisten das Lerner nur einen Raum gleichzeitig belegen können.
	\item Das System sollte die Möglichkeit bieten das Lehrkräfte mehrere Räume reservieren können.
	\item Das System sollte die Möglichkeit bieten das Lehrkräfte mehrere Räume belegen können.
	\item Das System sollte die Möglichkeit bieten das die Institut-Verwaltung mehrere Räume gleichzeitig reservieren kann.
	\item Das System sollte die Möglichkeit bieten das die Institut-Verwaltung mehrere Räume gleichzeitig belegen kann.
	\item Das System sollte die Möglichkeit bieten Benutzergruppen anhand von bestimmten Merkmalen zu unterscheiden.
	\item Das System sollte die Möglichkeit bieten Eingaben von unterschiedlichen Benutzergruppen zu verarbeiten. 
	\item Das System sollte die Möglichkeit bieten Benutzern eine Verlängerung der Raumbuchung nach Ablauf eines vordefinierten Zeitraums zu ermöglichen.
	\item Das System muss die Möglichkeit besitzen den aktuellen Standort des Benutzers zu bestimmen.
	\item Das System muss eine Überprüfung von aktuell vorhandenem Raumequipment innerhalb eines Raumes ermöglichen.
\end{itemize}

\subsubsection{Benutzerinteraktion}
\label{sec:Benutzerinteraktion}

\begin{itemize}
	\item Das System muss dem Benutzer die Möglichkeit bieten anhand von benutzerdefinierten Eingaben einen Raumvorschlag auszugeben.
	\item Das System muss dem Benutzer die Möglichkeit bieten sich als Mitglied einer Benutzergruppe zu verifizieren.
	\item Das System muss dem Benutzer die Möglichkeit bieten seine persönlichen Informationen zu verändern.
	\item Das System muss dem Benutzer die Möglichkeit bieten einen Raum zu buchen.
	\item Das System muss dem Benutzer die Möglichkeit bieten einen vorgegebenen Raum zu reservieren.
	\item Das System muss dem Benutzer die Möglichkeit bieten einen vorgegebenen Raum zu buchen.
	\item Das System muss dem Benutzer die Möglichkeit bieten die Reservierung eines Raumes aufheben.
	\item Das System muss dem Benutzer die Möglichkeit bieten die Buchung eines Raumes aufheben.
	\item Das System muss dem Benutzer die Möglichkeit bieten die Buchung eines Raumes zu verlängern.
	\item Das System sollte einem Administrator die Möglichkeit bieten neue Räume im System hinzuzufügen.
	\item Das System sollte einem Administrator die Möglichkeit bieten bestehende Rauminformationen zu aktuallisieren.
\end{itemize}

\subsubsection{Schnittstellenanforderungen}
\label{sec:Schnittstellenanforderungen}

\begin{itemize}
	\item Das System muss fähig sein, Informationen aus einem persistenten Speicher zu beziehen.
	\item Das System muss fähig sein, Informationen in einen persistenten Speicher zu schreiben.
	\item Das System muss fähig sein, Informationen vom Client zu beziehen.
	\item Das System muss fähig sein, Informationen an den Client zu senden.
	\item Das System muss fähig sein, eine Verbindung zu einem Netzwerk herzustellen.
	\item Das System muss fähig sein über ein Netzwerk mit anderen Systemkomponenten zu kommunizieren.
	\item Das System muss fähig sein Informationen auf Clientseite zu verarbeiten. 
	\item Das System muss Informationen in einem persistenten Datenspeicher eindeutig zuordnen können.
\end{itemize}


\subsection{Non-Funktionale Anforderungen}
\label{sec:Non-Funktionale_Anforderungen}

\subsubsection{Qualitäts Anforderungen}
\label{sec:Qualitäts_Anforderungen}

\begin{itemize}
	\item Das System muss dem Benutzer jederzeit die Korrektheit der präsentierten Informationen gewährleisten.
	\item Das System sollte dem Benutzer den Zugriff in Echtzeit auf die verfügbaren Informationen ermöglichen.
	\item Das System muss administrativ verwaltbar sein.
	\item Das System muss dem Benutzer eine einfache Konfiguration auf seinem Endgerät ermöglichen (weniger als 5 Arbeitsschritte).
	\item Das System muss dem Benutzer die Gültigkeit einer von ihm getätigten Interaktion garantieren.
	\item Das System sollte auf dem Endgerät des Benutzers maximal 100MB Speicherplatz benötigen.
	\item Die Schnittstelle des Systems muss für den Benutzer in einer portablen Form vorliegen.
	\item Das System muss für den Benutzer einfach zu erlernen sein.
	\item Das System sollte Fehlermeldungen eindeutig beschreiben.
	\item Das System sollte dem Benutzer alle benötigten Funktionen zur Aufgabenerledigung zur verfügung stellen.

	\item Das System sollte eine Benutzeroberfläche für administrative Arbeiten bereitstellen.   
	\item Das System muss Informationen über eine grafische Oberfläche ausgeben können.
	\item Das System muss Informationen über eine grafische Oberfläche einlesen können.
	\item Das System muss die Anfragen des Benutzers innerhalb von 10 Sekunden ausführen.
	\item Das System muss für den Benutzer innerhalb einer Applikation für ein mobiles Endgerät zur Verfügung stehen.
	\item Das System sollte Informationen im JSON-Format speichern können.
	\item Das System sollte das JSON-Format auslesen können.

	\item Das System muss eine verteilte Anwendungslogik besitzen.
	\item Das System muss 2 oder mehr Komponenten besitzen die auf unterschiedlichen Systemen laufen.
	\item Das System muss auf lokal betriebenen Rechnern ausgeführt werden können.
	\item Das System muss in den Sprachen Java oder JavaScript geschrieben werden.
	\item Das System muss von den Entwicklern des Projekts geschrieben werden. 
\end{itemize}

\subsubsection{Organisatorische Anforderungen}
\label{sec:Organisatorische_Anforderungen}

\begin{itemize}
	\item Das System muss bis zum 28.01.2018, 23:59 Uhr fertiggestellt sein.
\end{itemize}